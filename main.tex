\documentclass[modern]{aastex61}

\newcommand{\escapecmd}[1]{\texttt{\detokenize{#1}}}

\submitjournal{ApJ}

\shorttitle{Astropy Project II}
\shortauthors{Astropy Project et al.}

\begin{document}

\draft{\today}

\title{The Astropy Project: }

\correspondingauthor{Astropy coordination committee}
\email{coordinators@astropy.org}

\author{Astropy Collaboration}

\begin{abstract}
The Astropy project supports and fosters the development of open-source Python
packages that provide commonly-needed functionality to the astronomical
community.
A key element of the project is the Astropy core package, which serves as the
foundation for more specialized projects and packages.
In this article, we provide an overview of the organization of the Astropy
project and summarize key features in the core package as of the recent major
release, version 2.0.
We then describe the project infrastructure designed to facilitate [...]
We conclude with a future outlook of planned new features and directions for the
broader Astropy ecosystem.
\end{abstract}

%% Keywords should appear after the \end{abstract} command.
%% See the online documentation for the full list of available subject
%% keywords and the rules for their use.
\keywords{}

\section*{\textit{Notes and guidelines (to be removed)}}

\begin{itemize}
	\item We don't plan on including code in this paper, but if you think you will need to include code in your section, please add it to a separate Python module (.py file) and include it in this repository.
    \item Use \escapecmd{\sectionname} not ``Section,'' \escapecmd{\figurename} not ``Figure''
\end{itemize}

\section{Introduction} \label{sec:intro}
% Moritz (first draft)


\section{Major concepts}
% Anyone want to take this??

\subsection{Astropy development model, astropy ecosystem}
This subsection should include some basic statistics about the contributors, commits, line of code etc. up to the 2.0 release when we know it.

\subsection{Mixin}

\subsection{Accuracy testing across many different implementation}

\subsection{Use of Quantities across package}

%\subsection{Difficulty of reversing design choices}
%Deciding if a feature should be included
%difficulty to decide where general use ends and "handy feature for some" starts, i.e. how to reject PRs or deal with maintenance burden

\section{Astropy Core Package v2.0}

The astropy project aims to provide python-based packages for all tasks that are commonly needed in a large subset of the astronomical community. Two aspects of this (time and coordinate transformations) are already discussed in great detail in section 3. In this section, we highlight other features introduced or substantially improved since version v0.2, which is described in \citet{2013A&A...558A..33A}.
(ordered in the order in which they appear in the astropy documentation)

% \subsection{Analytic Functions}

\subsection{Constants}
% David S.
versioning

\subsection{Coordinates}
% Adrian, Erik

The \texttt{astropy.coordinates} subpackage is designed to support representing
and transforming celestial coordinates and velocities.


- Three-tiered structure for coordinates: representation, frame, skycoord
- With differential classes, frames support velocities

Accuracy and comparison of coordinate system definitions?

\textbf{Key features:}
\begin{itemize}
	\item Celestial coordinates, and local coordinates (via EarthLocation + Time)
	\item JPL ephemerids
\end{itemize}

\subsection{Units and quantities}
% Marten, Adrian

\textbf{Key features:}
\begin{itemize}
	\item incl logarithmic units and magnitudes
	\item speed improvements
    \item interaction with numpy arrays
\end{itemize}

\subsection{Data arrays}

\subsubsection{NDDdata}
% Matt C., Michael, or Steve C.

\subsubsection{Tables}
% Tom A.
QTable is new, mixin columns for time and coordinates. Table operations were added in v0.3

\subsection{Convolution}
% Adam G.

Astropy implements `normalized convolution' \citep[e.g.,][]{Knutsson1993}, which is an image reconstruction technique in which missing data are ignored during the convolution and replaced with values interpolated using the kernel.   In version $<=1.3$, the direct convolution and fft convolution approaches were not consistent, with fft convolution implementing normalized convolution and direct convolution implementing a different approach.  As of v2.0, the two methods are consistent and include a suite of consistency checks.


\begin{figure}
\includegraphics[width=\textwidth]{convolution_example.png}
An example showing different modes of convolution available in the python ecosystem.  The red x's mark pixels that are set to NaN in the original data (a).  If the data are convolved with a Gaussian kernel on a 9x9 grid using scipy's direct convolution (b), any pixel within range of the original NaN pixels is also set to NaN.  Panel (c) shows what happens if the NaNs are set to zero first: the originally NaN regions are depressed relative to their surroundings.  Finally, panel (d) shows astropy's convolution behavior, where the missing pixels are replaced with values interpolated from their surroundings using the convolution kernel.
\end{figure}


\subsection{Modeling}
% Nadia

The whole modeling submodule was missing from the previous paper, so everything really, including compound models, unit support etc.

\subsection{Visualization}
% Larry, Tom R.

wcsaxis, rgb, histograms comes to mind.Can use more publicity and also makes good images to include in the paper.

\subsection{Statistics}
% Steve C., Jake V.

major additions to be discussed: lomb-scargle, sigma clipping, bayesian blocks, histograms

\section{Infrastructure for affiliated packages}
\subsubsection{Package template}
bsipocz may write this up if there is no other takers
\subsection{Continuous integration helpers}
bsipocz should write this up
\subsection{sphinx extensions}
probably Tom R should write this up
\section{State of the Ecosystem}

\section{Learning Astropy}
% Kelle, Adrian

Explain components: Documentation, Example gallery, Tutorials and lessons

\section{The future of the Astropy project}

dropping python2 support, growths of affiliated packages
Summary

\acknowledgments

Who to thank?

%% Similar to \facility{}, there is the optional \software command to allow
%% authors a place to specify which programs were used during the creation of
%% the manusscript. Authors should list each code and include either a
%% citation or url to the code inside ()s when available.

\software{astropy \citep{2013A&A...558A..33A},
          numpy,
          scipy,
          }

%\bibliographystyle{aasjournal}
%\bibliography{bibliography}

\begin{thebibliography}{}

\end{thebibliography}



\end{document}

